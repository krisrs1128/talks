\documentclass{beamer}
\usetheme{Warsaw}
\usepackage{natbib}
\usepackage{graphicx}
\usepackage{graphicx}
\usepackage{libertine}
\usepackage{natbib}
\usepackage{bm}
\usepackage{subfig}
\renewcommand{\exp}[1]{\operatorname{exp}\left(#1\right)} % Exponential
\def\reals{\mathbb{R}} % Real number symbol
\def\naturals{\mathbb{N}} % Natural number symbol
\def\simplex{\mathcal{S}} % Simplex
\def\indic#1{\mbb{I}\left({#1}\right)} % Indicator function
\def\Gsn{\mathcal{N}}
\def\Bern{\text{Bern}}
\def\Dir{\text{Dir}}
\def\Mult{\text{Mult}}
\def\Unif{\textnormal{Unif}}
%\newcommand{\tabitem}{~~\llap{\textbullet}~~}
\newcommand{\tabitem}{\textbullet~~}

% shortcuts
\def\*#1{\mathbf{#1}}
\def\m#1{\boldsymbol{#1}}
\def\absarg#1{\left|#1\right|}

\usepackage{tabularx,ragged2e,booktabs,babel}
\newcolumntype{s}{>{\RaggedRight\hsize=.5\hsize}X}
\usepackage{csquotes}


\newcommand{\win}{\tilde w_{\text{in}}}
\newcommand{\wout}{\tilde w_{\text{out}}}


\AtBeginSection[]{
  \begin{frame}
  \vfill
  \centering
  \begin{beamercolorbox}[sep=8pt,center,shadow=true,rounded=true]{title}
    \usebeamerfont{title}\insertsectionhead\par%
  \end{beamercolorbox}
  \vfill
  \end{frame}
}

\usepackage{xcolor}
\usepackage{listings}

\definecolor{background}{HTML}{f7f7f7}
\definecolor{comment}{HTML}{67c582}
\definecolor{keyword}{HTML}{c08080}
\definecolor{number}{HTML}{8080c0}
\definecolor{string}{HTML}{8267c5}

\lstset{
  language=python,
  backgroundcolor=\color{background},
  commentstyle=\color{comment},
  keywordstyle=\color{keyword},
  numberstyle=\color{number},
  stringstyle=\color{string},
  basicstyle=\ttfamily\footnotesize
}

\setbeamersize{text margin left=5pt,text margin right=5pt}

\setbeamerfont{institute}{size=\fontsize{7pt}{8pt}}
\setbeamerfont{date}{size=\fontsize{0pt}{0pt}}

\title{Text Modeling meets the Microbiome}
\author{Kris Sankaran and Susan P. Holmes}
\institute{Department of Statistics, Stanford University}
\date{}
\begin{document}

\begin{frame}
  \maketitle
\begin{figure}
  \centering
  \includegraphics[width=0.5\textwidth]{figure/title.jpg}
\end{figure}

\end{frame}

\begin{frame}
  \frametitle{Microbiome vs. Text Analysis}
  \begingroup
  \fontsize{10pt}{10pt}\selectfont
  {\ttfamily
  \def\arraystretch{1.2}
  \setlength{\tabcolsep}{0.2em} % for the horizontal padding
  \begin{table}
    % latex table generated in R 3.3.2 by xtable 1.8-2 package
% Fri Sep  8 13:46:28 2017
\begin{table}[ht]
\centering
{\ttfamily
\def\arraystretch{1}
\setlength{\tabcolsep}{0.2em} % for the horizontal padding
\begin{tabular}{rlrrrrrrrr}
  \hline
passage & book & elizabeth & darcy & bennet & miss & jane & bingley & time & lady \\ 
  \hline
  0 & P \& P &   0 &   0 &   4 &   0 &   1 &   3 &   0 &   2 \\ 
    1 & P \& P &   1 &   0 &   5 &   0 &   1 &   4 &   0 &   0 \\ 
    2 & P \& P &   0 &   0 &   6 &   0 &   0 &   5 &   1 &   1 \\ 
    3 & P \& P &   1 &   4 &   5 &   1 &   0 &   9 &   1 &   3 \\ 
    4 & P \& P &   3 &   3 &   5 &   4 &   4 &   5 &   3 &   0 \\ 
    5 & P \& P &   3 &   0 &   0 &   2 &   1 &   6 &   1 &   0 \\ 
    6 & P \& P &   0 &   6 &   6 &   7 &   1 &   5 &   1 &   1 \\ 
    7 & P \& P &   1 &   3 &   1 &   5 &   3 &   2 &   1 &   0 \\ 
    8 & P \& P &   3 &   1 &   0 &   1 &   6 &   2 &   0 &   0 \\ 
    9 & P \& P &   2 &   6 &   0 &   3 &   0 &   0 &   0 &   1 \\ 
   10 & P \& P &   5 &   7 &   1 &   6 &   0 &   3 &   0 &   3 \\ 
   11 & P \& P &   1 &   0 &   3 &   2 &   0 &   1 &   1 &   0 \\ 
   12 & P \& P &   3 &   0 &   6 &   3 &   7 &   2 &   1 &   0 \\ 
   13 & P \& P &   9 &   1 &   2 &   3 &   4 &   3 &   0 &   0 \\ 
   14 & P \& P &   6 &   3 &   2 &   8 &   6 &   8 &   0 &   0 \\ 
   15 & P \& P &   4 &   3 &   1 &   3 &   0 &   5 &   2 &   0 \\ 
   16 & P \& P &   5 &   4 &   1 &   6 &   0 &   6 &   1 &   1 \\ 
   17 & P \& P &   3 &   0 &   7 &   6 &   3 &   8 &   2 &   1 \\ 
   18 & P \& P &   3 &   5 &   2 &   0 &   1 &   5 &   0 &   1 \\ 
   19 & P \& P &   5 &   6 &   3 &   4 &   3 &   7 &   1 &   1 \\ 
   \hline
\end{tabular}
}
\end{table}
 
    \\[12pt]
    % latex table generated in R 3.3.2 by xtable 1.8-2 package
% Fri Sep  8 14:31:53 2017
\begin{tabular}{rlrrrrr}
  \hline
time & subject & Unc06grq & Unc09fy6 & Unc06bhm & Unc06g1h & Unc06af7 \\ 
  \hline
  1 & D & 791 &   0 &  79 & 108 &  11 \\ 
    2 & D & 1616 &   0 & 1413 & 192 &  31 \\ 
    3 & D & 1323 &   0 & 915 & 165 &  23 \\ 
    4 & D & 1846 &   0 & 1366 & 170 &  31 \\ 
    5 & D & 2314 &   0 & 689 & 135 &  26 \\ 
    6 & D & 2244 &   0 & 776 & 310 & 175 \\ 
    7 & D & 1652 &   0 & 609 & 235 & 181 \\ 
    8 & D & 347 &   0 & 470 &  30 &   3 \\ 
    9 & D & 837 &   1 & 309 & 159 & 121 \\ 
   10 & D & 500 &   0 &  74 &  56 &  48 \\ 
   \hline
\end{tabular}
 
  \end{table}
  }
  \endgroup
\end{frame}

\begin{frame}
  \frametitle{Microbiome vs. Text Analysis}
 \begin{itemize}
 \item Each field could benefit by recognizing tools developed in the other
 \item Text analysis methods have been used successfully in the microbiome
   literature before \citep{cai2017learning, shafiei2015biomico,
     chen2012estimating, yan2017metatopics}
 \item Similarities
   \begin{itemize}
   \item Form: High-dimensional, sparse, count matrices, with supplemental
     information (``metadata'')
   \item Goals: Information-dense data compression
   \end{itemize}
 \item Differences
   \begin{itemize}
   \item Form: Sentences vs. networks
   \item Goals: Predictive black-boxes vs. scientific inference
   \end{itemize}
 \end{itemize} 
\end{frame}

\begin{frame}
  \frametitle{Microbiome vs. Text Analysis}
\begin{figure}[ht]
  \centering
  \includegraphics[width=1\textwidth]{figure/text_vs_microbiome}
  \caption{A translation, to help develop the analogy. \label{fig:label} }
\end{figure}

\end{frame}

\begin{frame}
  \frametitle{Example: Latent Dirichlet Allocation (LDA)}
\begin{itemize}
  \item Introduced in \citep{blei2003latent}
  \item Mixed-membership: Middle-ground between clustering (well-separated
    types) and ordination (continuous gradient)
  \item Predictive checks: Probabilistic formalism allows incorporation of
    study-specific structure and model assessment
\end{itemize}  
\begin{figure}[ht]
  \centering
  \includegraphics[width=0.37\textwidth]{figure/simplex}
  \caption{A toy representation of the mixed-membership idea.\label{fig:label} }
\end{figure}

\end{frame}

\begin{frame}
  \frametitle{Example: LDA}
 \begin{figure}[ht]
   \centering
   \includegraphics[width=0.9\textwidth]{figure/visualize_lda_theta_boxplot-F}\\
   \caption{The positions of points in the data of
     \citep{dethlefsen2011incomplete}. \label{fig:label} }
 \end{figure}
\end{frame}

\begin{frame}
  \frametitle{Example: LDA}
 \begin{figure}[ht]
   \centering
   \includegraphics[width=0.86\textwidth]{figure/visualize_lda_beta-F}
   \caption{An interpretation of the corners for the data of \citep{dethlefsen2011incomplete}. \label{fig:label} }
 \end{figure}
\end{frame}

\begin{frame}
  \frametitle{Model Assessment}
\begin{itemize}
\item Simulation and assessment allows iterative improvement
\item Poor fit can be used to identify interesting outliers
\end{itemize}
  \begin{figure}[!p]
    \centering
    \includegraphics[width=0.6\textwidth]{figure/posterior_check_ts-F}
    \caption{Posterior simulated trajectories for a subset of species, according
      to two models.}
  \end{figure} 
\end{frame}

\begin{frame}
  \frametitle{Conclusion}
 \begin{itemize}
 \item Distillation $>$ Addition: We have studied applications of common text
   analysis algorithms to the study microbiome-specific questions
 \item Reproducibility: All of our data and code is public
   \url{https://github.com/krisrs1128/microbiome_plvm/}
 \item Preprint: ``Latent Variable Modeling for the Microbiome''
   \url{https://arxiv.org/abs/1706.04969}
 \end{itemize} 
\end{frame}

\begin{frame}
\bibliographystyle{plainnat}
\bibliography{refs.bib}
\end{frame}

\end{document}
