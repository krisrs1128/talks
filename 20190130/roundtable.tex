\documentclass[10pt,mathserif]{beamer}

\usepackage{graphicx,amsmath,amssymb}
\usepackage{subcaption, natbib, hyperref}
\hypersetup{
    colorlinks=true,
    linkcolor=blue,
    filecolor=magenta,
    urlcolor=cyan,
}

\usepackage{graphicx}
\usepackage{libertine}
\usepackage{natbib}
\usepackage{bm}
\usepackage{subfig}
\renewcommand{\exp}[1]{\operatorname{exp}\left(#1\right)} % Exponential
\def\reals{\mathbb{R}} % Real number symbol
\def\naturals{\mathbb{N}} % Natural number symbol
\def\simplex{\mathcal{S}} % Simplex
\def\indic#1{\mbb{I}\left({#1}\right)} % Indicator function
\def\Gsn{\mathcal{N}}
\def\Bern{\text{Bern}}
\def\Dir{\text{Dir}}
\def\Mult{\text{Mult}}
\def\Unif{\textnormal{Unif}}
%\newcommand{\tabitem}{~~\llap{\textbullet}~~}
\newcommand{\tabitem}{\textbullet~~}

% shortcuts
\def\*#1{\mathbf{#1}}
\def\m#1{\boldsymbol{#1}}
\def\absarg#1{\left|#1\right|}

\usepackage{tabularx,ragged2e,booktabs,babel}
\newcolumntype{s}{>{\RaggedRight\hsize=.5\hsize}X}
\usepackage{csquotes}


\newcommand{\win}{\tilde w_{\text{in}}}
\newcommand{\wout}{\tilde w_{\text{out}}}


\AtBeginSection[]{
  \begin{frame}
  \vfill
  \centering
  \begin{beamercolorbox}[sep=8pt,center,shadow=true,rounded=true]{title}
    \usebeamerfont{title}\insertsectionhead\par%
  \end{beamercolorbox}
  \vfill
  \end{frame}
}

\usepackage{xcolor}
\usepackage{listings}

\definecolor{background}{HTML}{f7f7f7}
\definecolor{comment}{HTML}{67c582}
\definecolor{keyword}{HTML}{c08080}
\definecolor{number}{HTML}{8080c0}
\definecolor{string}{HTML}{8267c5}

\lstset{
  language=python,
  backgroundcolor=\color{background},
  commentstyle=\color{comment},
  keywordstyle=\color{keyword},
  numberstyle=\color{number},
  stringstyle=\color{string},
  basicstyle=\ttfamily\footnotesize
}


\mode<presentation>
{
\usetheme{default}
}
\setbeamertemplate{navigation symbols}{}
\usecolortheme[rgb={0.13,0.28,0.59}]{structure}
\setbeamertemplate{itemize subitem}{--}
\setbeamertemplate{frametitle} {
	\begin{center}
	  {\large\bf \insertframetitle}
	\end{center}
}

\AtBeginSection[]
{
	\begin{frame}<beamer>
		\frametitle{Outline}
		\tableofcontents[currentsection,currentsubsection]
	\end{frame}
}

%% begin presentation

\title{\large \bfseries Humanitarian AI Roundtable \\ Crisis Mapping and Super-Resolution}

\author{Kris Sankaran and Pablo Fonseca\\ (Work with Rifat Arefin, Anthony
  Ortiz, Jason Jo, Vincent Michalski, and Samira Kahou)}

\date{\today}

\begin{document}
\maketitle

%% \begin{frame}
%%   \frametitle{Maps are Useful}
%%   Especially if you have a critical mission.
%%  \begin{figure}[ht]
%%    \centering
%%    \includegraphics[width=0.6\paperwidth]{figures/old_map}
%%    \caption{A mapping of Tyrol, from the Second Military Survey of the Habsburg
%%      Empire.\label{fig:label} }
%%  \end{figure}
%% \end{frame}

%% \begin{frame}
%%   \frametitle{Maps are Useful}
%%  \begin{figure}[ht]
%%    \centering
%%    \includegraphics[width=0.6\paperwidth]{figures/napoleon_map}
%%    \caption{This is what false positives look like.}
%%  \end{figure}
%% \end{frame}

\begin{frame}
  \frametitle{Improving Sensing}
  \begin{itemize}
  \item Map Annotation
    \begin{itemize}
    \item How to augment human volunteers, and scale annotations?
    \item Use case: Pandemic response planning
    \end{itemize}
  \end{itemize}
  \begin{figure}
    \centering
    \includegraphics[width=0.3\paperwidth]{figures/mapathon}
    \caption{Exampling crowdsourcing session, from the MissingMaps
      websie. \label{fig:label} }
  \end{figure}
\end{frame}

\begin{frame}
  \frametitle{Research Problems: Map Annotation}
  \begin{itemize}
  \item \textbf{Conditional U-Net}: Models robust across a variety of environments
  \item \textbf{Interactive Corrections}: Leverage human volunteers efficiently
  \item \textbf{Useable Uncertainties}: Streamline validation and correction processes
  \item \textbf{Incremental Annotations}: Learn across a hierarchy of annotations
  \end{itemize}  
  \begin{figure}[ht]
    \centering
    \includegraphics[width=0.85\paperwidth]{figures/classical_interactive}
    \caption{One approach to more interactive image segmentation, from ``Deep
      Object Selection.'' \label{fig:label} }
  \end{figure}
\end{frame}

\begin{frame}
  \frametitle{Improving Sensing}
  \begin{itemize}
  \item Super-Resolution
    \begin{itemize}
    \item How to end monopolies on high-res maps?
    \item Use case: Quantifying extent of violence in Darfur
    \end{itemize}
  \end{itemize} 
  \begin{figure}[ht]
    \centering
    \includegraphics[width=0.4\paperwidth]{figures/high_res_khartoum}
    \caption{A high-resolution image of a street in Khartoum. \label{fig:label} }
  \end{figure}
\end{frame}

\begin{frame}
  \begin{itemize}
  \item Super-Resolution
    \begin{itemize}
    \item How to end monopolies on high-res satellite images?
    \item Use case: Quantifying extent of violence in Darfur
    \end{itemize}
  \end{itemize} 
  \begin{figure}[ht]
    \centering
    \includegraphics[width=0.8\paperwidth]{figures/multiframe_khartoum}
    \caption{Corresponding low-res views. \label{fig:label} }
  \end{figure}
\end{frame}

\begin{frame}
  \begin{itemize}
  \item Super-Resolution
    \begin{itemize}
    \item How to end monopolies on high-res satellite images?
    \item Use case: Quantifying extent of violence in Darfur
    \end{itemize}
  \end{itemize} 
  \begin{figure}[ht]
    \centering
    \includegraphics[width=0.7\paperwidth]{figures/multliframe_khartoum_2}
    \caption{Corresponding low-res views. \label{fig:label} }
  \end{figure}
\end{frame}

\begin{frame}
  \frametitle{Research Problems: Super-Resolution}
  \begin{itemize}
  \item Conditioning: How to incorporate metadata into super-resolution?
  \item Multiframedness: Dealing with alignment and using multiple inputs
    (Knowledge Graphs, multichannel, and recurrence)
  \end{itemize}
  \begin{figure}[ht]
    \centering
    \includegraphics[width=0.6\paperwidth]{figures/conditioning_network}
    \caption{Example of conditioning architecture used for super-resolution. \label{fig:label} }
\end{figure}


\end{frame}

\begin{frame}
  \frametitle{Challenges}
  \begin{itemize}
  \item Narrowing on problems -- schools or bridges? Myanmar or Uganda?
  \item Getting data!
  \item Abstractions vs. applications
  \end{itemize}
\end{frame}

\begin{frame}
  \frametitle{Lessons Learned}
  \begin{itemize}
  \item Decide on your own MNIST
  \item Define intermediate successes
  \item There are more venues than you think
  \end{itemize}
\end{frame}

\end{document}
