\documentclass{article}
\usepackage{natbib}
\usepackage{graphicx}
\usepackage{graphicx}
\usepackage{libertine}
\usepackage{natbib}
\usepackage{bm}
\usepackage{subfig}
\renewcommand{\exp}[1]{\operatorname{exp}\left(#1\right)} % Exponential
\def\reals{\mathbb{R}} % Real number symbol
\def\naturals{\mathbb{N}} % Natural number symbol
\def\simplex{\mathcal{S}} % Simplex
\def\indic#1{\mbb{I}\left({#1}\right)} % Indicator function
\def\Gsn{\mathcal{N}}
\def\Bern{\text{Bern}}
\def\Dir{\text{Dir}}
\def\Mult{\text{Mult}}
\def\Unif{\textnormal{Unif}}
%\newcommand{\tabitem}{~~\llap{\textbullet}~~}
\newcommand{\tabitem}{\textbullet~~}

% shortcuts
\def\*#1{\mathbf{#1}}
\def\m#1{\boldsymbol{#1}}
\def\absarg#1{\left|#1\right|}

\usepackage{tabularx,ragged2e,booktabs,babel}
\newcolumntype{s}{>{\RaggedRight\hsize=.5\hsize}X}
\usepackage{csquotes}


\newcommand{\win}{\tilde w_{\text{in}}}
\newcommand{\wout}{\tilde w_{\text{out}}}


\AtBeginSection[]{
  \begin{frame}
  \vfill
  \centering
  \begin{beamercolorbox}[sep=8pt,center,shadow=true,rounded=true]{title}
    \usebeamerfont{title}\insertsectionhead\par%
  \end{beamercolorbox}
  \vfill
  \end{frame}
}

\usepackage{xcolor}
\usepackage{listings}

\definecolor{background}{HTML}{f7f7f7}
\definecolor{comment}{HTML}{67c582}
\definecolor{keyword}{HTML}{c08080}
\definecolor{number}{HTML}{8080c0}
\definecolor{string}{HTML}{8267c5}

\lstset{
  language=python,
  backgroundcolor=\color{background},
  commentstyle=\color{comment},
  keywordstyle=\color{keyword},
  numberstyle=\color{number},
  stringstyle=\color{string},
  basicstyle=\ttfamily\footnotesize
}


\title{Discovery and Visualization of Latent Structure in the Microbiome}
\author{Kris Sankaran}

\begin{document}

Human microbiomes -- the collections of bacteria living around and within the
human body -- are complex ecological systems, and describing their structure and
function in different contexts is important from both basic scientific and
medical perspectives. Viewed through a statistical lens, many microbiome
analysis goals can framed in terms of discovering and describing latent
structure. For example, this structure might reflect sudden environmental shocks
that affect certain subsets of species, or may illuminate gradual shifts in
community composition. We will survey ideas from the data visualization and
probabilistic modeling literatures that we have found useful in identifying and
characterizing such structure in the microbiome. On the data visualization
front, we will describe the focus-plus-context and linking principles, and
introduce a new R package that uses these ideas to facilitate visualization of
hierarchies of time series. Then, motivated by the fact that microbiome species
abundance data often have effectively low-dimensional evolutionary, temporal,
and count structure, we explore the application of basic probabilistic latent
variable models, focusing on mixed-membership and matrix factorization
techniques. We illustrate and compare both visualization and modeling techniques
on a study of the effects of antibiotics on the human gut microbiome. Code and
data for all experiments is available publicly online.

\end{document}
