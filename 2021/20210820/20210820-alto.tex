\documentclass{beamer}
\usepackage{graphicx}
\usepackage{libertine}
\usepackage{natbib}
\usepackage{bm}
\usepackage{subfig}
\renewcommand{\exp}[1]{\operatorname{exp}\left(#1\right)} % Exponential
\def\reals{\mathbb{R}} % Real number symbol
\def\naturals{\mathbb{N}} % Natural number symbol
\def\simplex{\mathcal{S}} % Simplex
\def\indic#1{\mbb{I}\left({#1}\right)} % Indicator function
\def\Gsn{\mathcal{N}}
\def\Bern{\text{Bern}}
\def\Dir{\text{Dir}}
\def\Mult{\text{Mult}}
\def\Unif{\textnormal{Unif}}
%\newcommand{\tabitem}{~~\llap{\textbullet}~~}
\newcommand{\tabitem}{\textbullet~~}

% shortcuts
\def\*#1{\mathbf{#1}}
\def\m#1{\boldsymbol{#1}}
\def\absarg#1{\left|#1\right|}

\usepackage{tabularx,ragged2e,booktabs,babel}
\newcolumntype{s}{>{\RaggedRight\hsize=.5\hsize}X}
\usepackage{csquotes}


\newcommand{\win}{\tilde w_{\text{in}}}
\newcommand{\wout}{\tilde w_{\text{out}}}


\AtBeginSection[]{
  \begin{frame}
  \vfill
  \centering
  \begin{beamercolorbox}[sep=8pt,center,shadow=true,rounded=true]{title}
    \usebeamerfont{title}\insertsectionhead\par%
  \end{beamercolorbox}
  \vfill
  \end{frame}
}

\usepackage{xcolor}
\usepackage{listings}

\definecolor{background}{HTML}{f7f7f7}
\definecolor{comment}{HTML}{67c582}
\definecolor{keyword}{HTML}{c08080}
\definecolor{number}{HTML}{8080c0}
\definecolor{string}{HTML}{8267c5}

\lstset{
  language=python,
  backgroundcolor=\color{background},
  commentstyle=\color{comment},
  keywordstyle=\color{keyword},
  numberstyle=\color{number},
  stringstyle=\color{string},
  basicstyle=\ttfamily\footnotesize
}

%Information to be included in the title page:
\title{Multi-resolution Analysis of Count Data using Topic Alignment}
\author{Kris Sankaran}
\date{August 20, 2021}
%\graphicspath{{figure}}

\begin{document}

\frame{\titlepage}

\begin{frame}
  \frametitle{Preface}

  45 minutes: Topic Alignment
  \begin{itemize}
    \item How to turn this into a more widely-shareable talk?
    \item Which types of audiences to have in mind?
  \end{itemize}

  15 minutes: Soil microbiome study
  \begin{itemize}
  \item Data analysis suggestions?
  \end{itemize}
\end{frame}

\section{Topic Alignment}

\begin{frame}
  \frametitle{Scientific Problem}
  \begin{itemize}
  \item We are interested in what constitutes a healthy vaginal microbiome
  \item We want to understand risk factors that can be determined by analyzing
  \item To this end, a study has collected longitudinal microbiome data on X
  women over Y - Z weeks using 16S sequencing
  the microbiome
\end{itemize}
\end{frame}

\begin{frame}
  \frametitle{Data Analysis Problem}
\begin{itemize}
  \item We can assume that the raw sequencing reads have been passed through a
  pipeline that results in an $N$ samples by $D$ Amplicon Sequence Variant (ASV)
  count matrix
  \item We have a variety of characteristics about each mother, including
  whether or not they had a preterm birth
  \item The main sources of complexity are,
  \begin{itemize}
    \item Not all the known relevant bacterial species are known in advance
    \item There might be interaction effects that lead to improved / worsened
    outcomes
    \item $D$ is large
  \end{itemize}
\end{itemize}
\end{frame}
\begin{frame}
  \frametitle{Standard Approaches}
  \begin{itemize}
    \item Multiple Testing: Return a list of species that are strongly
    associated with preterm birth
    \item Hierarchical Clustering: Return a tree describing groups of species
    that tend to have similar abundance levels across samples
    \item Principal Component Analysis: Return a projection that shows how
    changes in species signature relates to variation in sample characteristics
  \end{itemize}
\end{frame}

\begin{frame}
  \frametitle{Why are these methods useful?}
    These outputs are cognitive artifacts that support scientific reasoning.
    Human working memory is limited, and these methods provide,
    \begin{itemize}
      \item A compressed representation that is easier to reason about
      (Clustering and PCA)
      \item Guidance of attention to details of interest (Multiple testing)
  \end{itemize}
\end{frame}

\begin{frame}
  \frametitle{Key Tasks}
  To solve problems like those that the VMRC cares about, it's necessary to
  assemble subtle clues into coherent, high-level theories. This requires both,
  \begin{itemize}
    \item Critical evaluation: What assumptions are actually being made about
    the data? It helps to have an explicit model.
    \item Navigation: We have a range of both simple and complex models easily
    available, but navigating the complexity-interpretability trade-offs between
    them is difficult
  \end{itemize}
\end{frame}

\begin{frame}
  \frametitle{Relevant Literature}
  \begin{itemize}
    \item Cluster comparison:
    \item Hierarchical Topic Models:
    \item Factored Topic Models:
  \end{itemize}
\end{frame}

\begin{frame}
  \frametitle{Main Idea}
  We will fit an ensemble of generative models, of varying levels of complexity,
  and each of which can be critically evaluated. Then, we will build a compact
  representation that streamlines navigation across them.
  (Provide a figure)
\end{frame}

\subsection{Methodology}

\begin{frame}
  \frametitle{Topic Models}
  On popular topic model is Latent Dirichlet Allocation (LDA), which supposes
  that each $x_{i}$ is drawn independently according to,
  \begin{align*}
  x_i \vert \gamma_i &\sim \Mult\left(n_{i}, B\gamma_{i}\right) \\
  \gamma_{i} &\sim \Dir\left(\lambda_{\gamma} 1_{K}\right)
  \end{align*}
  where the columns $\beta_{k}$ of $B \in \simplex^{D}$ lie in the $D$
  dimensional simplex and are themselves drawn independently from,
  \begin{align}
  \beta_{k} \sim \Dir\left(\lambda_{\beta} 1_{D}\right).
  \end{align}
  We will vertically stack the $N$ $\gamma_i$'s into an $N \times K$ matrix
  $\Gamma$.
\end{frame}

\begin{frame}
  \frametitle{Interpretation}
Topic models are well-suited to dimensionality reduction of count data. The
estimated parameters have the following interpretation,
\begin{itemize}
  \item $\Gamma \in \Delta_{K}^{N}$: Per-document memberships across $K$ topics.
  \item $B \in \Delta_{V}^{K}$: Per topic distributions over $V$ words.
\end{itemize}
\end{frame}

\begin{frame}
  \frametitle{Alignment as a Graph}

  We view an alignment as a graph across the model ensemble.

  \begin{itemize}
    $\{\beta^m_{k}, \gamma^m_{ik}\}$
  \end{itemize}

\end{frame}

\begin{frame}
  \frametitle{Summaries}
\end{frame}

\begin{frame}
  \frametitle{Key Topics}
\end{frame}


\begin{frame}
  \frametitle{Coherence}
\end{frame}

\begin{frame}
  \frametitle{Refinement}
\end{frame}

\subsection{Simulation}

\begin{frame}
\end{frame}

\subsection{Data Analysis}

\section{Analyzing THOR (The Hitchhikers of the Rhizosphere)}

\end{document}
